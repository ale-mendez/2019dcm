\documentclass[11pt]{article}
\usepackage{multicol}
\usepackage{graphicx}
\usepackage[spanish]{babel}
\usepackage[utf8]{inputenc}

% This is the LaTeX sample. If you use this sample you automatically get 
%the correct formatting of your Abstract.
\voffset=0.0cm
\hoffset=0.0cm
\topskip=0cm
\topmargin=-2.0cm
\oddsidemargin=-0.5cm
\evensidemargin=-0.5cm
\textwidth=16.75cm \textheight=25.0cm
\renewcommand\refname{\small References}
\renewcommand\baselinestretch{1}
\pagestyle{empty}
\newcommand{\abstracttitle}[1]{\begin{center}#1\end{center}}
\newcommand{\authors}[1]{\vspace*{-0.3cm}\begin{center}#1\end{center}\vspace*{-0.3cm}}
\newcommand{\addresses}[1]{\begin{center}#1\end{center}}
\newcommand{\abstracttext}[1]{\vspace{0.5cm}\columnsep0.75cm \begin{multicols}{2} #1 \end{multicols}}
% Use \picturelportrait{0} when you want to include a portrait figure!
\newcommand{\pictureportrait}[2]{\vspace{0.5cm}\centerline{\includegraphics*[width=7.8cm,height=9cm,angle=#1]{#2}}}
% Use \picturelandscape{0} when you want to include a landscape figure!
\newcommand{\picturelandscape}[2]{\vspace{0.5cm}\centerline{\includegraphics*[width=7.8cm,height=5.5cm,angle=#1]{#2}}}
\newcommand{\capt}[2]{\vspace*{-0.1cm}\begin{center} \begin{minipage}[t]{7.0cm} Fig. #1. #2 \end{minipage}\end{center}\vspace*{0.3cm}}
\newcommand{\captTab}[2]{\vspace*{-0.1cm}\begin{center} \begin{minipage}[t]{7.0cm} Tab. #1. #2 \end{minipage}\end{center}\vspace*{-0.1cm}}


\begin{document}


\abstracttitle{IMPLEMENTACIÓN DE MÉTODOS DE APRENDIZAJE AUTOMATIZADO
\\ EN PROBLEMAS COLISIONALES}

\authors{
A M P Mendez$^\ast$,
J I Di Filippo$^\dag$,
S D López$^\ast$
and D M Mitnik$^{\ast,\dag}$ 
}

% \abstracttitle{}%end \abstracttitle
% \authors{


\addresses{
$^\ast$Instituto de Astronom\'ia y F\'isica del Espacio, Consejo 
Nacional de Investigaciones Cient\'ificas y T\'ecnicas and 
Universidad de Buenos Aires, Buenos Aires, Argentina \\
$^\dag$Departamento de F\'isica, Universidad de Buenos Aires, 
Buenos Aires, Argentina
}


\abstracttext{
Las herramientas desarrolladas en el campo del {\it machine
learning} resultan de gran aplicabilidad en problemas de 
la física que exigen constante intervención humana. 
En el presente reporte mostramos algunos ejemplos de 
dicha implementación en la física de colisiones. 
En particular, utilizamos herramientas de aprendizaje automatizado 
para describir con precisión blancos atómicos en procesos
colisionales.

Los cálculos de transiciones inelásticas requieren la 
representación de los estados ligados y continuos involucrados. 
En principio, la existencia de un potencial efectivo 
que describa dichos estados permitiría obtener de forma directa 
las funciones de onda de las partículas interactuantes.
El método de inversión depurada (DIM)~\cite{Mendez:16} permite
la obtención de potenciales efectivos orbitales atómicos y
moleculares. El método consiste en la inversión directa de 
ecuaciones de tipo Kohn--Sham, cuyas soluciones están dadas 
por las ecuaciones de Hartree--Fock. El potencial resultante,
plagado de polos y/o divergencias, es ajustado a través de una
expresión analítica paramétrica de manera tal que las energías
y valores medios originales son reproducidos con gran precisión. 
Sin embargo, el ajuste del potencial y optimización de los
parámetros es un proceso ``artesanal'' que demanda recursos 
humanos. 
% La inclusión de métodos automatizados en dicha técnica significa 
% una notable mejoría en el método, reduciendo drásticamente los
% niveles de intervención humana y proporcionando incluso mejores
% resultados en los ajustes analíticos.

Por otro lado, métodos sotisticados tales como la expansión 
de {\em close-cloupling} requieren una descripción precisa del 
blanco~\cite{Bartschat:04,Zatsarinny:16}. 
Sin embargo, la determinación de una estructura 
atómica adecuada a menudo requiere experiencia y recursos 
computacionales significativos. En general, la función de onda del 
blanco se expresa 
mediante el método de {\em configuration interaction} (CI). Además,
la parte radial de las funciones de onda de un electrón se puede 
obtener con potenciales modelo que contienen parámetros de 
escaleo~\cite{Badnell:11}.
La precisión de la estructura se puede mejorar aumentando el número 
de configuraciones en el CI, lo que a su vez aumenta el número de
parámetros que se pueden variar. Sin embargo, no existe una receta
sistemática ni lógica para este procedimiento. 
% Hemos demostrado que la implementación de métodos de aprendizaje
% automatizado en este tipo de problemas reduce significativamente 
% el costo de cálculo en la mayoría de los casos, lo cual resulta 
% determinante en iones de centenas de niveles.

La inclusión de herramientas de aprendizaje automatizado en la 
optimización de potenciales y funciones de onda constituye una 
mejora significativa en ambos métodos. A partir de su 
implementación, hemos logrado reducir drásticamente los niveles 
de intervención y recursos humanos, proporcionando incluso 
mejores resultados.



\begin{thebibliography}{99}
\bibitem{Mendez:16} 
A.M.P. Mendez, D.M. Mitnik, and J.E. Miraglia, 
Int. J. Quant. Chem. {\bf 116}, 1882 (2016); 
Advances in Quantum Chemistry \textit{accepted} (2019).

\bibitem{Bartschat:04} 
Bartschat, K {\em et al} J. Phys. B {\bf 37} 2617 (2004).

\bibitem{Zatsarinny:16}
Zatsarinny, O {\em et al} 
J. Phys. B {\bf 49} 235701 (2016).

\bibitem{Badnell:11}
Badnell, N R 
Comput. Phys. Commun. {\bf 7} 1528 (2011).

\end{thebibliography}

% \noindent
% [5] The GPyOpt authors 2016 \url{http://github.com/SheffieldML/GPyOpt} 
% 
% \noindent
% [6] J\"onsson, P {\em et al} 1999 {\em J. Phys. B} {\bf 32} 1233 
% 
% \noindent
% [7] Kramida, A {\em et al} NIST Atomic Spectra Database (version 5.6.1)  

}%\abstracttext

% end of the body
\end{document}
